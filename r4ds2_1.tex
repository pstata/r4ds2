% Options for packages loaded elsewhere
\PassOptionsToPackage{unicode}{hyperref}
\PassOptionsToPackage{hyphens}{url}
\PassOptionsToPackage{dvipsnames,svgnames,x11names}{xcolor}
%
\documentclass[
  letterpaper,
  DIV=11,
  numbers=noendperiod]{scrartcl}

\usepackage{amsmath,amssymb}
\usepackage{lmodern}
\usepackage{iftex}
\ifPDFTeX
  \usepackage[T1]{fontenc}
  \usepackage[utf8]{inputenc}
  \usepackage{textcomp} % provide euro and other symbols
\else % if luatex or xetex
  \usepackage{unicode-math}
  \defaultfontfeatures{Scale=MatchLowercase}
  \defaultfontfeatures[\rmfamily]{Ligatures=TeX,Scale=1}
\fi
% Use upquote if available, for straight quotes in verbatim environments
\IfFileExists{upquote.sty}{\usepackage{upquote}}{}
\IfFileExists{microtype.sty}{% use microtype if available
  \usepackage[]{microtype}
  \UseMicrotypeSet[protrusion]{basicmath} % disable protrusion for tt fonts
}{}
\makeatletter
\@ifundefined{KOMAClassName}{% if non-KOMA class
  \IfFileExists{parskip.sty}{%
    \usepackage{parskip}
  }{% else
    \setlength{\parindent}{0pt}
    \setlength{\parskip}{6pt plus 2pt minus 1pt}}
}{% if KOMA class
  \KOMAoptions{parskip=half}}
\makeatother
\usepackage{xcolor}
\setlength{\emergencystretch}{3em} % prevent overfull lines
\setcounter{secnumdepth}{-\maxdimen} % remove section numbering
% Make \paragraph and \subparagraph free-standing
\ifx\paragraph\undefined\else
  \let\oldparagraph\paragraph
  \renewcommand{\paragraph}[1]{\oldparagraph{#1}\mbox{}}
\fi
\ifx\subparagraph\undefined\else
  \let\oldsubparagraph\subparagraph
  \renewcommand{\subparagraph}[1]{\oldsubparagraph{#1}\mbox{}}
\fi


\providecommand{\tightlist}{%
  \setlength{\itemsep}{0pt}\setlength{\parskip}{0pt}}\usepackage{longtable,booktabs,array}
\usepackage{calc} % for calculating minipage widths
% Correct order of tables after \paragraph or \subparagraph
\usepackage{etoolbox}
\makeatletter
\patchcmd\longtable{\par}{\if@noskipsec\mbox{}\fi\par}{}{}
\makeatother
% Allow footnotes in longtable head/foot
\IfFileExists{footnotehyper.sty}{\usepackage{footnotehyper}}{\usepackage{footnote}}
\makesavenoteenv{longtable}
\usepackage{graphicx}
\makeatletter
\def\maxwidth{\ifdim\Gin@nat@width>\linewidth\linewidth\else\Gin@nat@width\fi}
\def\maxheight{\ifdim\Gin@nat@height>\textheight\textheight\else\Gin@nat@height\fi}
\makeatother
% Scale images if necessary, so that they will not overflow the page
% margins by default, and it is still possible to overwrite the defaults
% using explicit options in \includegraphics[width, height, ...]{}
\setkeys{Gin}{width=\maxwidth,height=\maxheight,keepaspectratio}
% Set default figure placement to htbp
\makeatletter
\def\fps@figure{htbp}
\makeatother

\KOMAoption{captions}{tableheading}
\makeatletter
\makeatother
\makeatletter
\makeatother
\makeatletter
\@ifpackageloaded{caption}{}{\usepackage{caption}}
\AtBeginDocument{%
\ifdefined\contentsname
  \renewcommand*\contentsname{Table of contents}
\else
  \newcommand\contentsname{Table of contents}
\fi
\ifdefined\listfigurename
  \renewcommand*\listfigurename{List of Figures}
\else
  \newcommand\listfigurename{List of Figures}
\fi
\ifdefined\listtablename
  \renewcommand*\listtablename{List of Tables}
\else
  \newcommand\listtablename{List of Tables}
\fi
\ifdefined\figurename
  \renewcommand*\figurename{Figure}
\else
  \newcommand\figurename{Figure}
\fi
\ifdefined\tablename
  \renewcommand*\tablename{Table}
\else
  \newcommand\tablename{Table}
\fi
}
\@ifpackageloaded{float}{}{\usepackage{float}}
\floatstyle{ruled}
\@ifundefined{c@chapter}{\newfloat{codelisting}{h}{lop}}{\newfloat{codelisting}{h}{lop}[chapter]}
\floatname{codelisting}{Listing}
\newcommand*\listoflistings{\listof{codelisting}{List of Listings}}
\makeatother
\makeatletter
\@ifpackageloaded{caption}{}{\usepackage{caption}}
\@ifpackageloaded{subcaption}{}{\usepackage{subcaption}}
\makeatother
\makeatletter
\@ifpackageloaded{tcolorbox}{}{\usepackage[many]{tcolorbox}}
\makeatother
\makeatletter
\@ifundefined{shadecolor}{\definecolor{shadecolor}{rgb}{.97, .97, .97}}
\makeatother
\makeatletter
\makeatother
\ifLuaTeX
  \usepackage{selnolig}  % disable illegal ligatures
\fi
\IfFileExists{bookmark.sty}{\usepackage{bookmark}}{\usepackage{hyperref}}
\IfFileExists{xurl.sty}{\usepackage{xurl}}{} % add URL line breaks if available
\urlstyle{same} % disable monospaced font for URLs
\hypersetup{
  pdftitle={r4ds2},
  pdfauthor={Phanikumar S Tata},
  colorlinks=true,
  linkcolor={blue},
  filecolor={Maroon},
  citecolor={Blue},
  urlcolor={Blue},
  pdfcreator={LaTeX via pandoc}}

\title{r4ds2}
\author{Phanikumar S Tata}
\date{}

\begin{document}
\maketitle
\ifdefined\Shaded\renewenvironment{Shaded}{\begin{tcolorbox}[boxrule=0pt, frame hidden, sharp corners, interior hidden, borderline west={3pt}{0pt}{shadecolor}, breakable, enhanced]}{\end{tcolorbox}}\fi

\hypertarget{introduction}{%
\section{Introduction}\label{introduction}}

The Introduction gives an overview of what the book covers.
\textbf{Learning objectives:}

\begin{itemize}
\tightlist
\item
  Describe a \textbf{typical data science project.}
\item
  Explain the reasoning behind the \textbf{order of content in this
  book.}
\item
  Recognize topics that are explicitly \textbf{not covered by this
  book.}
\item
  \textbf{Set up an environment} in which you can learn the topics in
  this book.
\end{itemize}

\hypertarget{introduction-1}{%
\subsection{Introduction}\label{introduction-1}}

\begin{itemize}
\tightlist
\item
  Describe how \textbf{code in the book} differs from \textbf{code in
  your console.}
\item
  Recall ways to \textbf{get help with R code.}

  \begin{itemize}
  \tightlist
  \item
    Produce a minimal reproducible example or \textbf{reprex.}
  \end{itemize}
\end{itemize}

\hypertarget{the-order-of-content-in-this-book}{%
\section{The order of content in this
book}\label{the-order-of-content-in-this-book}}

\begin{itemize}
\tightlist
\item
  Import \& Tidy are boring, so we jump to \textbf{visualization \&
  transformation.}
\item
  After that we learn to \textbf{wrangle (import \& tidy) data,} because
  that is a necessary skill.
\item
  Those baseline skills enables us to start \textbf{programming.}
  Learning to program helps us simplify the other steps.
\item
  We might then get into \textbf{modeling} and \textbf{communicating,}
  or we might pick those up in books that are more specifically devoted
  to those skills.
\end{itemize}

\hypertarget{the-order-of-bcd1rlg-would-read-for-this}{%
\section{The order of BCD1RLG would read for
this}\label{the-order-of-bcd1rlg-would-read-for-this}}

\begin{itemize}
\tightlist
\item
  we will jump to \textbf{Import \& Tidy.}
\item
  After that we learn to \textbf{wrangle (import \& tidy) data,} because
  that is a necessary skill.
\item
  Then to \textbf{visualization \& transformation.}
\item
  Those baseline skills enables us to start \textbf{programming.}
  Learning to program helps us simplify the other steps.
\end{itemize}

\hypertarget{not-covered-by-this-book}{%
\subsection{Not covered by this book}\label{not-covered-by-this-book}}

\begin{itemize}
\tightlist
\item
  \textbf{Big data:} Working with big data is problem-specific. If you
  need to work with big data, other tools will be useful to learn.
\item
  \textbf{Python, Julia, etc:} This book focuses on R. Master one tool
  at a time, but maybe go on to other tools later.
\item
  \textbf{Non-rectangular data:} Honestly even a lot of things that
  aren't naturally table-like can be coerced to be table-like, so it
  makes sense to start with tables.
\item
  \textbf{Hypothesis confirmation:} This book focuses on exploratory
  data analysis.
\end{itemize}

\hypertarget{setting-up-an-environment}{%
\subsection{Setting up an environment}\label{setting-up-an-environment}}

We'll need:

\begin{itemize}
\tightlist
\item
  \textbf{R}
\item
  \textbf{Rstudio}
\item
  The \textbf{tidyverse} (\texttt{install.packages("tidyverse")})
\item
  Three \textbf{additional packages}
  (\texttt{install.packages(c("nycflights13",\ "gapminder",\ "Lahman"))})
\end{itemize}

\hypertarget{running-r-code}{%
\subsection{Running R code}\label{running-r-code}}

\begin{itemize}
\tightlist
\item
  Code in the book has some slight differences from code on your
  console.
\item
  Don't freak out.
\end{itemize}



\end{document}
